\documentclass[a4paper]{article}
\usepackage{amsmath}
\usepackage[margin=1in]{geometry}
\title{Notatki Algebra i Geometria Liniowa}
\author{Mateusz Kojro}
\begin{document}
\maketitle
    \section{Wiadomosci wstepne}
    \subsection{Zbiory}
    \paragraph{Def Zbior pusty:}
    zbior ktory nie zawiera zadnego elementu oznaczamy \begin{equation}
        A = \emptyset
    \end{equation}
    \paragraph{Def Podzbior:}
    Mowimy ze A jest podzbiorem B jezeli:
    \begin{equation}
        (A \subset B) \Leftrightarrow \forall a(a \in A \Rightarrow a \in B)
    \end{equation}
    \paragraph{Def Zbiory rowne}
    Zbiory sa rowne jezeli:
    \begin{equation}
        (A = B) \Leftrightarrow \forall a (a \in A \Leftrightarrow a \in B)
    \end{equation}
    \paragraph{Def Suma Zbiorow}
    Suma zbiorow A i B nazywamy
    \begin{equation}
        A \cup B = \{ x : x \in A \vee x \in B\}
    \end{equation}
    \paragraph{Def Iloczyn zbiorow}
    Iloczyn zbiorow A i B nazywamy:
    \begin{equation}
        A \cap B = \{ x:x \in A \wedge x \in B \}
    \end{equation}
    \paragraph{Def Roznica zbiorow}
    Roznica zbiorow nazywamy:
    \begin{equation}
        A - B  = \{x:x \in A \wedge x \notin B\}
    \end{equation}
    \paragraph{Def Alternatywa rozlaczna (XOR) chyba}
    \begin{equation}
        A \div B = \{ x : (x \in A \wedge x \notin B ) \vee (x \notin A \wedge x \notin B) \}
    \end{equation}
    \paragraph{Def Iloczyn Kartezjanski}
    Iloczynem Kartezjanskim zbiorow nazywamy
    \begin{equation}
        A_1 \times A_2 \times \ldots \times A_n = \{ (a_1,a_2, \ldots ,a_n ) : a_i \in A_i, i = 1,2,\ldots,n \}
    \end{equation}
\subsection{Odwzorowania}
\paragraph{Def Odwzorowanie zbioru A w zbior B}
kazdemu elemntowi a z zbioru A przyporzadkujemy dokladnie jeden element b z B oznaczamy:
\begin{equation}
h : A \rightarrow B
\end{equation}
Gdzie:
\begin{itemize}
    \item Dziedzina : A
    \item Przeciwdziedzina odwzorowania A (obraz zbioru A) : $ h(A) \subset B $
    \item Przeciwobraz zbioru $ B_1 $ : $ A_1 = h^{-1}(B_1) $ taki ze $ h(A_1) \subset B_1 $
\end{itemize}
\paragraph{Def Superpozycja (zlozenie)}
zlozeniem odwzorowan $ h : A \rightarrow B $ i $ g : B \rightarrow C $ nazywamy
\begin{equation}
    g \circ h : A \rightarrow C
\end{equation}
 takie ze
\begin{equation}
    (g \circ h)(a) = g(h(a)) \ \forall_{a \in A}
\end{equation}
\paragraph{Def Iniekcja (odwzorowanie roznowartosciowe)}
$ h:A \rightarrow B $ jest Iniekcja gdy:
\begin{equation}
    \forall_{a_1,a_2 \in A} \ h(a_1) = h(a_2) \Rightarrow a_1 = a_2
\end{equation}
\paragraph{Def Surjekcja (odwzorowanie na)}
$ h:A \rightarrow B $ jest Surjekcja gdy:
\begin{equation}
    \forall_{b \in B} \exists_{a \in A } \ h(a) = b
\end{equation}
\paragraph{Def Bijekcja (odwzorowanie wzajemnie jednoznaczne)}
Odwzorowanie jest Bijekcja jezeli jest Iniekcja i Surjekcja
\paragraph{Def Odwzorowanie odwrotne}
jezeli $ h:A \rightarrow B $ jest Bijekcja to Odwzorowaniem odwrotnym nazywamy $ h^{-1}:B \rightarrow A $ takie ze
\begin{equation}
\forall_{a \in A} (h^{-1} \circ h)(a) = a
\end{equation}
inaczej
\begin{equation}
    h^{-1} \circ h = Id_A
\end{equation}
\paragraph{Def Identycznosc}
Odwzorowanie zbioru A w siebie w postaci $ Id_A(a) =a $
\section{Struktury Algebraiczne}
\subsection{Grupy}
\paragraph{Def Zamknietosc wzoru wzgledem dzialania $ \oplus $}
Zbior A jest zamkniety wzgledem $ \oplus $ (dzialanie $\oplus$ jest wykonalne w zbiorze A) jezeli:
\begin{equation}
    \forall_{a,b \in A} \exists_{c \in A} c = a \oplus b
\end{equation}
Przyklady:
\begin{itemize}
    \item Zbior $N$  \underline{nie jest} jest zamkniety wzgledem odejmowania
    \item Zbior liczb calkowitych $Z$ \underline{jest} zamkniety wzgledem  + i * ale \underline{nie jest} zamkniety wzgledem $\div$ bo wynik dzielenia liczb calkowitych moze nie byc liczba calkowita
\end{itemize}
\paragraph{Grupa}
Grupa nazywamy pare zbioru G z dzialaniem $\circ$ wzgledem ktorego zbior jest zamkniety jezeli spelnione sa aksjomaty grupy:
\begin{itemize}
    \item Prawo lacznosci: $ \forall_{a,b,c \in G} $ \ $ a \circ (b \circ c) = (a \circ b) \circ c $
    \item Istnienie elementu neutralnego: $\exists_{e \in G} \forall_{a \in G}$ \ $ a \circ e = e \circ a = a $
    \item Istnienie element odwrotnego: $ \forall_{a \in G}$ \ $ \exists_{a^{-1} \in G} $ \ $ a \circ a^{-1} = a^{-1} \circ a = e $
\end{itemize}
\paragraph{Grupa Abelowa (przemienna)}
Grupe nazywamy Abelowa jezeli jej dzialanie jest przemiene:
\begin{equation}
    \forall_{a,b \in G} \ a \circ b = b \circ a
\end{equation}
\paragraph{Przyklady}
\ldots
\subsection{Ciała}
\paragraph{Ciało}
Cialem nazywamy zbior $K$ zawierajacy wiecej niz jeden element i niech bedzie zamkniety wzgledem dwoch dzialan $ (\oplus, \circ) $ jezeli zachodza relacje:
\begin{itemize}
    \item Przemiennosc dodawania: $ a \oplus b = b \oplus a $
    \item Przemiennosc mnozenia: $ a \circ b = b \circ a $
    \item Lacznosc dodawania: $ a \oplus (b \oplus c) = (a \oplus b) \oplus c $
    \item Lacznosc mnozenia: $ a \circ (b \circ c) = (a \circ b) \circ c $
    \item Rozdzielnosc $\circ$ wzgledem $\oplus$: \ $a \circ (b \oplus c) = a \circ b \oplus a \circ c$
    \item Istnienie zera: $ \exists!_{\theta \in K} \forall_{a \in  K} \ a \oplus \theta$ = a
    \item Wykonalnosc odejmowania: $ \forall_{a,b \in K} \exists!_{c \in K}$ \ $ a \oplus c = b $
    \item Wykonalnosc dzielenia: $\forall_{a,b \in K \ \text{i} \ a \ne \theta} \ \exists!_{c \in K} \ a \circ c = b$
\end{itemize}
W zwiazku z tym struktura $ (K, \oplus ,\circ) $ jest cialem wzgledem dzialan $ (\oplus, \circ) $ \ $ \Leftrightarrow $ struktury $ (K, \oplus) \ \text{i} \ (K, \circ) $ sa grupami abelowymi z rodzielnoscia $\circ$ wzgledem $\oplus$
\subsection{Pierścienie}
\paragraph{Def Pierscien (nieprzemienny)}
Pierścieniem nieprzemiennym nazywamy zbior P bedacy grupa Abelowa wzgledem dzialania dodawania $\oplus$ o elemencie neutralnym $\theta$ w którym spelnione jest Prawo lacznosci mnozenia $\circ$ przy czym zachodza oba prawa rodzielnosci
\begin{equation}
    a \circ (b \oplus c) = a \circ b \oplus a \circ \ c \ \text{oraz} \ (a \oplus b) \circ c = a \circ c \oplus b \circ c
\end{equation}
\paragraph{Def Pierscien (przemienny)}
jezeli w zbiorze P zachodzi takze prawo przemiennosci to pierscien jest przemienny
\section{Cialo liczb zespoloncyh}
\subsection{Definicja}
Cialo liczb zespoloncyh:
\begin{equation}
    C=(R\times R, +', * )
\end{equation}
\subsection{Postac algebraiczna}
Kazda liczbe zespolona $ z = (a,b) \in C $ mozemy przedstawic  w postacialgebraicznej
\begin{equation}
z = a + b * i , \ i = \sqrt{-1}
\end{equation}
\subsection{Postac trygonometryczna}
Licze zespolona $z = a + bi \ne 0$ moze byc zapisana w postaci geometrycznej:
\begin{equation}
    z = |z|(cos(\phi_0) + i\sin(\phi_0)) \ \text{gdzie} \ \phi_0 = \text{Arg}(z)
\end{equation}
\paragraph{Wzor de Moivre'a}
jezeli $    z = |z|(cos(\phi_0) + i\sin(\phi_0))$ to:
\begin{equation}
    z^{n} = |z|^{n}(\cos(n\phi)+ i\sin(n\phi)) \ \forall_{n \in N}
\end{equation}
\paragraph{Pierwiastki Zespolone}
jezeli $    z = |z|(cos(\phi_0) + i\sin(\phi_0))$ oraz $w \in C$, to rownanie $w^{n} = z$ ma dokladnie n rozwiazan:
\begin{equation}
    w = z_{k} \ (k = 0,1, \ldots , n -1)
\end{equation}
pierwiastkow zespoloncyh stopnia $n$ z liczby zespolonej $z$ gdzie:
\begin{equation}
z_{k} = \sqrt[n]{|z|}(\cos((\phi + 2k\pi)/n)+ i sin((\phi + 2k\pi)/n))
\end{equation}
\subsection{Interpretacja geometryczna}
%TODO
Do zrobienia \ldots
\subsection{Postac Eulera liczby zespolonej}
\begin{equation}
    z = |z|e^{i\phi}, \text{gdzie} \ \phi = \text{arg}(z)
\end{equation}
\end{document}