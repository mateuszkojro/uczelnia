\documentclass[a4paper]{article}
\usepackage{amsmath}
\usepackage[margin=1in]{geometry}
\title{Notatki Algebra i Geometria Liniowa}
\author{Mateusz Kojro}
\begin{document}
\maketitle
%
% Wiadomosci wstepne
%
    \section{Wiadomosci wstepne}
    \subsection{Zbiory}
    \paragraph{Def Zbior pusty:}
    zbior ktory nie zawiera zadnego elementu oznaczamy \begin{equation}
        A = \emptyset
    \end{equation}
    \paragraph{Def Podzbior:}
    Mowimy ze A jest podzbiorem B jezeli:
    \begin{equation}
        (A \subset B) \Leftrightarrow \forall a(a \in A \Rightarrow a \in B)
    \end{equation}
    \paragraph{Def Zbiory rowne}
    Zbiory sa rowne jezeli:
    \begin{equation}
        (A = B) \Leftrightarrow \forall a (a \in A \Leftrightarrow a \in B)
    \end{equation}
    \paragraph{Def Suma Zbiorow}
    Suma zbiorow A i B nazywamy
    \begin{equation}
        A \cup B = \{ x : x \in A \vee x \in B\}
    \end{equation}
    \paragraph{Def Iloczyn zbiorow}
    Iloczyn zbiorow A i B nazywamy:
    \begin{equation}
        A \cap B = \{ x:x \in A \wedge x \in B \}
    \end{equation}
    \paragraph{Def Roznica zbiorow}
    Roznica zbiorow nazywamy:
    \begin{equation}
        A - B  = \{x:x \in A \wedge x \notin B\}
    \end{equation}
    \paragraph{Def Alternatywa rozlaczna (XOR) chyba}
    \begin{equation}
        A \div B = \{ x : (x \in A \wedge x \notin B ) \vee (x \notin A \wedge x \notin B) \}
    \end{equation}
    \paragraph{Def Iloczyn Kartezjanski}
    Iloczynem Kartezjanskim zbiorow nazywamy
    \begin{equation}
        A_1 \times A_2 \times \ldots \times A_n = \{ (a_1,a_2, \ldots ,a_n ) : a_i \in A_i, i = 1,2,\ldots,n \}
    \end{equation}
\subsection{Odwzorowania}
\paragraph{Def Odwzorowanie zbioru A w zbior B}
kazdemu elemntowi a z zbioru A przyporzadkujemy dokladnie jeden element b z B oznaczamy:
\begin{equation}
h : A \rightarrow B
\end{equation}
Gdzie:
\begin{itemize}
    \item Dziedzina : A
    \item Przeciwdziedzina odwzorowania A (obraz zbioru A) : $ h(A) \subset B $
    \item Przeciwobraz zbioru $ B_1 $ : $ A_1 = h^{-1}(B_1) $ taki ze $ h(A_1) \subset B_1 $
\end{itemize}
\paragraph{Def Superpozycja (zlozenie)}
zlozeniem odwzorowan $ h : A \rightarrow B $ i $ g : B \rightarrow C $ nazywamy
\begin{equation}
    g \circ h : A \rightarrow C
\end{equation}
 takie ze
\begin{equation}
    (g \circ h)(a) = g(h(a)) \ \forall_{a \in A}
\end{equation}
\paragraph{Def Iniekcja (odwzorowanie roznowartosciowe)}
$ h:A \rightarrow B $ jest Iniekcja gdy:
\begin{equation}
    \forall_{a_1,a_2 \in A} \ h(a_1) = h(a_2) \Rightarrow a_1 = a_2
\end{equation}
\paragraph{Def Surjekcja (odwzorowanie na)}
$ h:A \rightarrow B $ jest Surjekcja gdy:
\begin{equation}
    \forall_{b \in B} \exists_{a \in A } \ h(a) = b
\end{equation}
\paragraph{Def Bijekcja (odwzorowanie wzajemnie jednoznaczne)}
Odwzorowanie jest Bijekcja jezeli jest Iniekcja i Surjekcja
\paragraph{Def Odwzorowanie odwrotne}
jezeli $ h:A \rightarrow B $ jest Bijekcja to Odwzorowaniem odwrotnym nazywamy $ h^{-1}:B \rightarrow A $ takie ze
\begin{equation}
\forall_{a \in A} (h^{-1} \circ h)(a) = a
\end{equation}
inaczej
\begin{equation}
    h^{-1} \circ h = Id_A
\end{equation}
\paragraph{Def Identycznosc}
Odwzorowanie zbioru A w siebie w postaci $ Id_A(a) =a $
\clearpage
%
% Struktury algebraiczne
%
\section{Struktury Algebraiczne}
\subsection{Grupy}
\paragraph{Def Zamknietosc wzoru wzgledem dzialania $ \oplus $}
Zbior A jest zamkniety wzgledem $ \oplus $ (dzialanie $\oplus$ jest wykonalne w zbiorze A) jezeli:
\begin{equation}
    \forall_{a,b \in A} \exists_{c \in A} c = a \oplus b
\end{equation}
Przyklady:
\begin{itemize}
    \item Zbior $N$  \underline{nie jest} jest zamkniety wzgledem odejmowania
    \item Zbior liczb calkowitych $Z$ \underline{jest} zamkniety wzgledem  + i * ale \underline{nie jest} zamkniety wzgledem $\div$ bo wynik dzielenia liczb calkowitych moze nie byc liczba calkowita
\end{itemize}
\paragraph{Grupa}
Grupa nazywamy pare zbioru G z dzialaniem $\circ$ wzgledem ktorego zbior jest zamkniety jezeli spelnione sa aksjomaty grupy:
\begin{itemize}
    \item Prawo lacznosci: $ \forall_{a,b,c \in G} $ \ $ a \circ (b \circ c) = (a \circ b) \circ c $
    \item Istnienie elementu neutralnego: $\exists_{e \in G} \forall_{a \in G}$ \ $ a \circ e = e \circ a = a $
    \item Istnienie element odwrotnego: $ \forall_{a \in G}$ \ $ \exists_{a^{-1} \in G} $ \ $ a \circ a^{-1} = a^{-1} \circ a = e $
\end{itemize}
\paragraph{Grupa Abelowa (przemienna)}
Grupe nazywamy Abelowa jezeli jej dzialanie jest przemiene:
\begin{equation}
    \forall_{a,b \in G} \ a \circ b = b \circ a
\end{equation}
\paragraph{Przyklady}
\ldots
\subsection{Ciała}
\paragraph{Ciało}
Cialem nazywamy zbior $K$ zawierajacy wiecej niz jeden element i niech bedzie zamkniety wzgledem dwoch dzialan $ (\oplus, \circ) $ jezeli zachodza relacje:
\begin{itemize}
    \item Przemiennosc dodawania: $ a \oplus b = b \oplus a $
    \item Przemiennosc mnozenia: $ a \circ b = b \circ a $
    \item Lacznosc dodawania: $ a \oplus (b \oplus c) = (a \oplus b) \oplus c $
    \item Lacznosc mnozenia: $ a \circ (b \circ c) = (a \circ b) \circ c $
    \item Rozdzielnosc $\circ$ wzgledem $\oplus$: \ $a \circ (b \oplus c) = a \circ b \oplus a \circ c$
    \item Istnienie zera: $ \exists!_{\theta \in K} \forall_{a \in  K} \ a \oplus \theta$ = a
    \item Wykonalnosc odejmowania: $ \forall_{a,b \in K} \exists!_{c \in K}$ \ $ a \oplus c = b $
    \item Wykonalnosc dzielenia: $\forall_{a,b \in K \ \text{i} \ a \ne \theta} \ \exists!_{c \in K} \ a \circ c = b$
\end{itemize}
W zwiazku z tym struktura $ (K, \oplus ,\circ) $ jest cialem wzgledem dzialan $ (\oplus, \circ) $ \ $ \Leftrightarrow $ struktury $ (K, \oplus) \ \text{i} \ (K, \circ) $ sa grupami abelowymi z rodzielnoscia $\circ$ wzgledem $\oplus$
\subsection{Pierścienie}
\paragraph{Def Pierscien (nieprzemienny)}
Pierścieniem nieprzemiennym nazywamy zbior P bedacy grupa Abelowa wzgledem dzialania dodawania $\oplus$ o elemencie neutralnym $\theta$ w którym spelnione jest Prawo lacznosci mnozenia $\circ$ przy czym zachodza oba prawa rodzielnosci
\begin{equation}
    a \circ (b \oplus c) = a \circ b \oplus a \circ \ c \ \text{oraz} \ (a \oplus b) \circ c = a \circ c \oplus b \circ c
\end{equation}
\paragraph{Def Pierscien (przemienny)}
jezeli w zbiorze P zachodzi takze prawo przemiennosci to pierscien jest przemienny
\clearpage
%
% Liczby zespolone
%
\section{Cialo liczb zespoloncyh}
\subsection{Definicja}
Cialo liczb zespoloncyh:
\begin{equation}
    C=(R\times R, +', * )
\end{equation}
\subsection{Postac algebraiczna}
Kazda liczbe zespolona $ z = (a,b) \in C $ mozemy przedstawic  w postacialgebraicznej
\begin{equation}
z = a + b * i , \ i = \sqrt{-1}
\end{equation}
\subsection{Wlasnosci}
\begin{itemize}
    \item jezeli $z = a+bi$ to $|z| = \sqrt{a^{2}+b^{2}}$
    \item jezeli $z = a+bi$ to $\phi = arg(z)$ jezeli \begin{itemize}
        \item $cos{\phi} = \frac{a}{|z|}$
        \item $sin{\phi} = \frac{b}{|z|}$
        \item jezeli $\phi < 2\pi$ to argument glowny ozn $Arg(z)$
    \end{itemize}
\end{itemize}
\subsection{Postac trygonometryczna}
Licze zespolona $z = a + bi \ne 0$ moze byc zapisana w postaci geometrycznej:
\begin{equation}
    z = |z|(cos(\phi_0) + i\sin(\phi_0)) \ \text{gdzie} \ \phi_0 = \text{Arg}(z)
\end{equation}
\paragraph{Wzor de Moivre'a}
jezeli $    z = |z|(cos(\phi_0) + i\sin(\phi_0))$ to:
\begin{equation}
    z^{n} = |z|^{n}(\cos(n\phi)+ i\sin(n\phi)) \ \forall_{n \in N}
\end{equation}
\paragraph{Pierwiastki Zespolone}
jezeli $    z = |z|(cos(\phi_0) + i\sin(\phi_0))$ oraz $w \in C$, to rownanie $w^{n} = z$ ma dokladnie n rozwiazan:
\begin{equation}
    w = z_{k} \ (k = 0,1, \ldots , n -1)
\end{equation}
pierwiastkow zespoloncyh stopnia $n$ z liczby zespolonej $z$ gdzie:
\begin{equation}
z_{k} = \sqrt[n]{|z|}(\cos((\phi + 2k\pi)/n)+ i sin((\phi + 2k\pi)/n))
\end{equation}
\subsection{Interpretacja geometryczna}
%TODO
Do zrobienia \ldots
\subsection{Postac Eulera liczby zespolonej}
\begin{equation}
    z = |z|e^{i\phi}, \text{gdzie} \ \phi = \text{arg}(z)
\end{equation}
\clearpage
\section{Iloczyn Skalarny}
\subsection{Iloczyn Skalarny}
Przyporzadkowanie $V \times V \ni \ \rightarrow <x,y> \ \in K$ nazywamy Iloczynem Skalarnym wektorow x , y
\paragraph{Przestrzen Euklidesowa (Unitarna)}
natomiast to para $(V,<*,*>)$ nad cialem skalarow $K$
\paragraph{jezeli:} $V$ bedzie przestrzenia wektorowa nad cialem K $(K \in R , K \in C \ldots)$ kazdej parze wektorow $x, y$ przyporzadkujemy skalar $< x, y >$ taki ze:
\begin{itemize}
    \item $ <x, y > = <y,x> $
    \item $ < \alpha x, y> = \alpha \ , \ \forall_{\alpha \in K} $
    \item $ <x + x', y> = <x,y> + <x' , y>$
    \item $<x,x> \ge 0$ , przy czym $<x,x> = 0 \Leftrightarrow x = 0$
\end{itemize}

\paragraph{Przy czym wyrozniamy formy: - uzupelnic warunki form}
\begin{itemize}
    \item Forma dwuliniowa symetryczna (hermitowska)
    \item Forma kwadratowa
    \item Forma kwadratowa dodatnio okreslona
\end{itemize}
Dlatego iloczynem Skalarnym nazywa sie dowolna forme dwuliniowa symetryczna lub hermitowska dodatnio okreslona
\subsection{Twierdzenia}
\paragraph{Tweirdzenie Sylwestra:}Macierz kwadratowa $A$ jest dodatnio okreslona jezeli wszystkie jej minory glownwe ostatnich elemntow glownej przekatnej sa dodatnie

\paragraph{Def Slad Macierzy M ($Tr(M)$)} jest suma jej elementow diagonalnych $m_{ii}$
\paragraph{Def Długosc wektora} jezeli $x \in (V, <*,*>)$ to dlugosc wektora to:
\begin{equation}
    |x| = \sqrt{<x,x>}
\end{equation}
\paragraph{Def Kat dwoch wektorow} jezeli wektory $x,y \in E = (V,<*,*>)$ to $\phi$ jest katem okreslonym:
\begin{equation}
    \phi = \arccos(\frac{<x,x>}{|x||x|})
\end{equation}
\paragraph{Def Wektory ortagonalne} wektory $ x,y\in (V, <*,*>) $ nazywamy ortagonalnymi jezeli:
\begin{equation}
    <x,y> = 0
\end{equation}
jezeli wektor jest zerowy to jest ortagonalny do kazdego w $E$
\paragraph{Def Odleglosc wektorow}
Odlegloscia wektorow $x,y \in E = (V, <*,*>)$ nazywamy liczbe d:
\begin{equation}
    d = |x -y |
\end{equation}
\paragraph{Def jakis dziwny wniosek o wektorach}
\paragraph{Def Unormowanie niezerowgo wektora} to operacja na $x \in E = (V,<*,*>)$ nazywamy operacje dazaca do utworzenia wersora:
\begin{equation}
    \hat{x} = \frac{x}{|x|} \ \text{,taki ze } \ |\hat{x}| = 1
\end{equation}
\paragraph{Twierdzenie Pitagorasa (w ogolnych przestrzeniach wektorowych)} dla ortagonalnych parami wektorow $x,y,z,\ldots ,w\in E=(V, <*,*>)$ zachodzi:
\begin{equation}
    |x+y+z+\ldots +w|^{2} = |x|^{2} + |y|^{2} + |z|^{2} + \ldots + |w|^{2}
\end{equation}
\paragraph{Nierownosc Schwartza}
dla dowolnych wektorow $x,y \in E = (V,<*,*>)$ zachodzi:
\begin{equation}
    <x,y>^{2} \ \le \ <x,x> <y,y>
\end{equation}
Z nierownosci Schwartza wynika nierownosc trojkata
\begin{equation}
    |x+y|\le|x|+|y|
\end{equation}
\paragraph{Ortagonalne bazy przestrzeni Euklidesa}
Niezerowe wektory $\{b_{1}, b_{2}, \ldots ,b_{n}\}$ n wymiarowej przestrzeni Euklidesowej $E = (V,<*,*>)$ tworza baze zwana:
\begin{itemize}
    \item Ortagonalna: jezeli sa parami prostopadle
    \item Orto-normalna: gdy sa parai prostopadle i sa wersorami
\end{itemize}
mamy wtedy:
\begin{equation}
    <b_{k}, b_{p}> = \delta_{kp} \ (k,p = 1,2, \ldots , n)
\end{equation}
\subparagraph{Wnioski:}

\begin{itemize}
    \item W bazie ortagonalnej $\{b_{1}, b_{2}, \ldots ,b_{n}\}$ wspolrzedne wektora $x_{k}$ wektora $x$ sa w postaci:
    \begin{equation}
       (*) \ \ x_{k} = <x, b_{k}> / <b_{k}, b_{k}>
    \end{equation}
    i tworza rzuty wektora $x$ na kolejne wektory bazy
    \item W przestrzeni Euklidesowej z baza orto-normalna Iloczyn Skalarnyjest w postaci standardowej:
    \begin{equation}
        <x,y> = x_1 y_1 + x_2 y_2 + x_3 y_3 + \ldots + x_n y_n
    \end{equation}
    \subparagraph{} gdzie
    $$n = dim V$$
    Istotnie Wektory $x$ i $y$ sa liniowymi kombinacjami wektorow bazy i teza wynika z liniowosci iloczynu skalarnego wobec relacji $(*)$
\end{itemize}


\paragraph{Algorytm ortagonalizacji Grama-Schmidta}
Kazda $n$ wymiarowa przestrzen Euklidesowa posiada baze Ortagonalna
\subparagraph{Wnioski:}
\begin{itemize}
    \item Wiersze dowolnej macierzy ortagonalnej stopnia $k$ stanowia baze ortonormalna przestrzeni wektorowej $R^{k}$
    \item Wiersze macierzy kwadratowej stopnia $k$ ktora spelnia $AA^{T} = D$ gdzie macierz D jest pewna macierza diagonalna stanowia baze ortagonalna przestrzeni wektorowej $R^{k}$
\end{itemize}
\section{Przestrzen Euklidesowa}
\paragraph{Wektor ortagonalny do p.p.w.}
Jezeli $ E = (V, <*,*>) $ bedzie przestrzenia euklidesowa, a $ V_{1}$
podprzestrzenia wektorowa przestrzeni $V$ to wektor $h \in V$ nazywamy wektorem ortogonalnym do podprzestrzeni przestrzeni wektorowej $V_1$, jeśli jest on ortogonalny do każdego wektora
\begin{equation}
    y \subset V1: <h, y> = 0
\end{equation}
\subparagraph{Wnioski}
Aby wektor $h\in V$ byl ortagonalny do podprzestrzeni przestrzeni wektorowej $V_1$ wystarczy zeby byl ortagonalny do wszystkich  wektorow dowolnej bazy przestrzeni wektorowej $V_1$ -- kompletnie tego nie ogarniam (wyklad 2b str 1)

\paragraph{Def Podprzestrzenie ortagonalne} $V_1$ i $V_2$ w przestrzeni Euklidesowej $E = (V,<*,*>)$ nazywamy ortagonalnymi jezeli dowolne:
\begin{equation}
    x \ in V_1 \ , \ y \in V_2
\end{equation}
sa Ortagonalne
\paragraph{Def Dopelnienie ortagonalne podprzestrzeni przestrzeni wektorowej}
to zbior wszystkich wektorow przestrzeni wektorowej V ortagonalnych do podprzestrzeni przestrzeni wektorowej $V_1 \subset V $ oznaczamy:
\begin{equation}
    V_1^{\perp}
\end{equation}
\paragraph{Def Rzut wektora na podprzestrzen przetrzedni wektorowej}
jezeli $V_1$ jest p.p.w przestrzeni Euklidesowej E = (V, <*,*>) i dla kazdego wektora $y \notin V_1$ wektor $y_0 \in V_1$ taki ze wektor $h = y - y_0$ jest ortagonalny do $V_1$ to wektor $y_0$ jest rzutem wektora $y$ na $V_1$
\clearpage
\section{Wektory}
\section{Macierze}
\subsection{Dodawanie (odejmowanie)}
\begin{equation}
    \begin{bmatrix}
        a & b & c\\
        d & e & f
        \end{bmatrix} +
        \begin{bmatrix}
            x & y & z \\
            q & r & t
        \end{bmatrix} =
        \begin{bmatrix}
            a+x & b+y & c +z \\
            d+q & e+r & f+t
        \end{bmatrix}
\end{equation}

\subsection{Mnozenie (dzieleine)}
Mozliwe tylko wtedy kiedy $Kolumny (1) = Wiersze(2)$
\begin{equation*}
    \begin{bmatrix}
        a & b & c\\
        d & e & f
        \end{bmatrix} \times
        \begin{bmatrix}
            x & y \\
            z & q \\
            r & t
        \end{bmatrix} =
        \begin{bmatrix}
            a*x + b*z + c *r & a*y + b * q + c * f  \\
            d*x + e * z + f *r & d* y + e * q + f*t
        \end{bmatrix}
\end{equation*}
\subsection{Wyznacznik Macierzy}
oznaczamy:
\begin{equation}
    det A \ \text{lub} \ |A|
\end{equation}
obliczamy:
\begin{itemize}
    \item $n = 2$
    \begin{equation}
    \begin{bmatrix}
        a_{11} & a_{12} \\
        a_{21} & a_{22}
    \end{bmatrix} = a_{11}a_{22} - a_{21}a_{12}
    \end{equation}
    \item $n = 3$
    \begin{equation}
        \begin{bmatrix}
            a_{11} & a_{12} & a_{13}\\
            a_{21} & a_{22} & a_{23} \\
            a_{31} & a_{32} & a_{33}
        \end{bmatrix} = a_{11}a_{22}a_{33} + a_{12}a_{23}a_{31} + \\ a_{13}a_{21}a_{32} - a_{11}a_{23}a_{32} - a_{12}a_{21}a_{33} - a_{13}a_{22}a_{31}
    \end{equation}
    \item $n>3$ Rozwiniecie Laplacea:
    Dla macierzy
    $A =
    \begin{bmatrix}
            a_{11} & a_{12} & a_{13}\\
            a_{21} & a_{22} & a_{23} \\
            a_{31} & a_{32} & a_{33}
    \end{bmatrix} $
    wybieramy rzad w ktorym najwiecej $(0,1,-1)$
    \begin{equation}
        det A = \sum^{n}_{j=1}a_{ij}\times detA_{ij}
    \end{equation}
    gdzie $i$ jest wybranym przez nas rzedem i potem liczymy wyznaczniki mniejszych macierzy wynikowych a kazdy z nich mnozymy przez $ a_{ij} * (-1)^{j+j} $
\end{itemize}
\paragraph{ Macierz kwadratowa ktorej wyznacznik jest rowny $ 0 $ nazywamy $Osobliwa$
}\subsection{Rzad macierzy}
Rzad macierzy to najwyzszy stopien dla ktorego wyznacznik jest rozny od zera \\
np jezeli wyznacznik macierzy $4 \times 4$ nie jest rowy $0$ to liczymy wszystkie wyznaczniki podmacierzy $3 \times 3$ poki nie znajdziemy takiego roznego od $0$ jezeli dalej takiego nie bedzie robimy tak samo dla $2 \times 2$ itd.
\subsection{Rozwiazywanie rownan macierzowych}
\end{document}
