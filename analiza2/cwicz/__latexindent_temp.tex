\documentclass[]{article}
\usepackage{amsmath}

\title{Analiza cwiczenia podsumowanie}
\author{Mateusz Kojro}
\begin{document}
\maketitle
\section{Definicja calki}
\subsection{Funkcja pierwotna funkcji f nazywamy:}
\begin{equation}
  \forall_{x \in (a,b)} F'(x) = f(x)
\end{equation}
\subsection{Definicja calki}
\begin{equation}
  \int f(x)dx = F(x) + C
\end{equation}
\section{Metody obliczania calek}
\subsection{Calkowanie przez czesci}
Jezeli funkcje zmiennej $x$: $a(x)$ i $b(x)$ maja pochodne ciagle to:
\begin{equation}
  \int{a(x) b'(x) dx} = a(x)b(x) - \int{b(x)a'(x)dx}
\end{equation}
lub zapisane inaczej:
\begin{equation}
  \int{adb} = ab - \int{bda}
\end{equation}
\subsection{Calkowanie przez podstawianie}
\label{subsec:podstawianie}
\subsubsection{Def:}
Jezeli:
\begin{itemize}
\item Dla $x \in [a,b]$ funkcja $g(x)$ ma pochodna ciagla
\item $f(x)$ jest ciagla w zbiorze wartosci $g(x)$
\end{itemize}
To mamy:
\begin{equation}
  \int{f(g(x)) \times g'(x)dx} = \int{f(u)du}, \ u = g(x)
\end{equation}
\subsection{Rozwiazanie}

\begin{equation}
  \Int{sin(4x)} =
  \begin{cases}
    u  & = 4x \\
    du & = 4dx& \\
    dx & = \frac{du}{4}&
  \end{cases}
\end{equation}
  \begin{equation}
    \sgn(x) \eq
    \begin{cases}
      -1 & \text{if } x < 0 \\
      0  & \text{if } x = 0 \\
      1  & \text{if } x > 0
    \end{cases}
  \end{equation}






\end{document}
