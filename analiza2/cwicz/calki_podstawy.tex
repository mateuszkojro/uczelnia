\documentclass[]{article}
\usepackage{amsmath}
\title{Analiza cwiczenia podsumowanie - calki podstawy}
\author{Mateusz Kojro}
\begin{document}
\maketitle
\section{Calka oznaczona}
\subsection{Funkcja pierwotna funkcji f nazywamy:}
\begin{equation}
  \forall_{x \in (a,b)} F'(x) = f(x)
\end{equation}
\subsection{Definicja calki}
\begin{equation}
  \int f(x)dx = F(x) + C
\end{equation}
\section{Metody obliczania calek}
\subsection{Calkowanie przez czesci}
Jezeli funkcje zmiennej $x$: $a(x)$ i $b(x)$ maja pochodne ciagle to:
\begin{equation}
  \int{a(x) b'(x) dx} = a(x)b(x) - \int{b(x)a'(x)dx}
\end{equation}
lub zapisane inaczej:
\begin{equation}
  \int{adb} = ab - \int{bda}
\end{equation}
\subsection{Calkowanie przez podstawianie}
\label{subsec:podstawianie}
\subsubsection{Def:}
Jezeli:
\begin{itemize}
\item Dla $x \in [a,b]$ funkcja $g(x)$ ma pochodna ciagla
\item $f(x)$ jest ciagla w zbiorze wartosci $g(x)$
\end{itemize}
To mamy:
\begin{equation}
  \int{f(g(x)) \times g'(x)dx} = \int{f(u)du}, \ u = g(x)
\end{equation}
\subsection{Rozwiazywanie}

\begin{align}
  \int{sin(4x)} &=
  \begin{vmatrix}
    u =4x\\
    du=4dx\\
    dx= \frac{du}{4}
  \end{vmatrix}
  \text{, bo \ } df=f'(x)\times dx \\
  \int{\sin{u}\frac{du}{4}} &= \frac{1}{4} \int{\sin{u}du} \\
  &= \frac{1}{4} \times (-\cos{u})\text{, }u=4x \\
  &= \frac{1}{4} \times (-\cos{4x}) 
\end{align}

\section{Calka oznaczona}
\label{sec:calka oznaczona}
\subsection{Definicja}
\label{subsec:Definicja calka oznaczona}
Jezeli dla $x=[a,b]$ wartosci funkcji $f(x) \ge 0$ to pole $P$ ogarniczone liniami $x = a$ i $x = b$ oraz wykresem $f(x)$ jest rowne calce oznaczonej:
\begin{equation}
P = \int_{a}^{b}f(x)dx
\end{equation}
i zachodzi:
\begin{align}
  \int_{a}^{b}{f(x)dx} &= F(b) - F(a) \\
  &= F(x) |^b_{a}
\end{align}
jezeli natomiast dla $x \in [a,b]$ $f(x) \le 0$:
\begin{equation}
  \label{}
P = -\int_a^b{f(x)dx}
\end{equation}
\subsection{Wzory:}
\par
Dla calek oznaczonych dzialaja wzory zwiazane z calkami nieoznaczonymi i dodatkowo:
\par
Jezeli mamy przedzial calkowania $[a,b]$ mozemy go podzielic na mniejsze przedzialy i policzyc 2 osobne calki po czym je dodac
\begin{align}
  [a,b] &= [a,c] + [c,b] \\
  \int_a^b{f(x)} &= \int_a^c{f(x)} + \int_c^b{f(x)} 
\end{align}
\section{Calki f. nieograniczonych}
Jezeli:
\begin{itemize}
\item $f(x)$ jest nieograniczona w przedziale $[a,b]$
\item jest ograniczona i calkowalna w przedzialach:
  \begin{align}
    a &\le x \le c - \alpha \\
    c + \beta &\le x \le b    
  \end{align}
gdzie:
  \begin{equation}
    c &\in [a,b] \ \text{, i }\alpha , \beta &> 0
  \end{equation}
\end{itemize}
To mamy:
 \begin{align}
    \int_a^bf(x)dx = \lim_{\alpha \to 0} \int_{a}^{c-\alpha}f(x)dx +  \lim_{\beta \to 0} \int_{c+\beta}^{b}f(x)dx 
  \end{align}
 \section{Calki oznaczone na przedzialach nieskonczonych}
 \label{sec:label}
 Mamy:
 \begin{equation}
   \label{}
   \int_a^{+ \infty}f(x)dx = \lim_{b \to + \infty} \int_{a}^{b}f(x)dx

 \end{equation}
 analogicznie:
 \begin{equation}
 \label{}
\int_{-\infty}^{+ \infty}f(x)dx = \lim_{b \to + \infty} \lim_{a \to + - \infty} \int_{a}^{b}f(x)dx
 \end{equation}
 \clearpage
\section{Calki funkcji wymiernych}
Jezeli mamy calke fukcji wymiernej w postaci:
\begin{equation}
  \int \frac{W_1}{W_2} = \int \frac{\sum_{i = 0}^n a_i \times x^i}{\sum_{j = 0}^m a_j \times x^j}dx
\end{equation}
to jezeli
\begin{itemize}
\item $n \ge m$ - dzielimy wielomiany przez siebie i otrzymujemy wielomian i wymierna gdzie $n \le m$
\item $n \le m$ - rozkladamy na sume ulamkow prostych:
  \begin{align}
    \frac{A}{(ax+b)^q} , \frac{Bx+C}{(cx^2+dx+e)^r} \text{ \ gdzie \ }
    \begin{cases}
      A,B,C,a,b,c,d,e &\in R \\
      q,r &\in N
    \end{cases}
  \end{align}
  i $cx^2+dx+e$ nie ma pierwiastkow i $\Delta \le 0$
\end{itemize}
\end{document}
