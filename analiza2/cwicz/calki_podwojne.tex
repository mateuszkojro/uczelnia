\documentclass[options]{article}
\author{Mateusz Kojro}
\usepackage{amsmath}
\title{Analiza cwiczenia podsumowanie - calki podwojne}
\begin{document}
\maketitle
\section{Definicja:}
\subsection{Def ogulna}
\label{subsec:label}


Jezeli mamy funkcje 2 zmiennych $f(x,y)$ i: 
\begin{itemize}
\item $f$ jest ciagla w $A$
\item $f$ jest ograniczona w $A$
\item $A$ jest w dziedzinie $f$
\end{itemize}
to calke podwojna oznaczamy:
\begin{equation}
  \label{}
  \int \int_Af(x,y)dxdy
\end{equation}
\subsection{Interpretacja geometryczna}
Jezeli $ \forall_{x,y \in A}$ $f(x,y) > 0 $ to :
\begin{equation}
  \label{}
V = \int \int_{A} f(x,y)dxdy
\end{equation}
\clearpage
\section{Obliczanie:}
\subsection{Obszar normalny}
Obszar normalny wzgledem osi $OX$:
\begin{align}
  (x,y) =
  \begin{cases}
    a &\le x \le b \\
    g(x) &\le y \le h(x)
  \end{cases}
\end{align}
Obszar normalny wzgledem osi $OY$:
\begin{align}
  (x,y) =
  \begin{cases}
    c &\le y \le d \\
    p(y) &\le x \le r(y)
  \end{cases}
 \end{align}
\section{Zamiana calki podwojnej na iterowana}
Jezeli $A$ jest normalna wzgledem $OX$:
\begin{equation}
\int \int_{A} f(x,y)dxdy = \int_a^b \bigg (\int_{y=g(x)}^{y=h(x)}f(x,y)dy \bigg )dx
\end{equation}
Jezeli $A$ jest normalna wzgledem $OY$:
\begin{equation}
\int \int_{A} f(x,y)dxdy = \int_c^d \bigg (\int_{x=p(y)}^{x=r(y)}f(x,y)dx \bigg )dy
\end{equation}
jezeli natomiast obszar calkowania jest dany nierownosciami:
\begin{equation}
  \label{}
a \le x \le b \text{  i  } c \le y \le d
\end{equation}
to:
\begin{align}
  \label{}
  \int \int_A f(x,y) dxdy &= \int_c^d \bigg (\int_a^b f(x,y)dx \bigg )dy \\
  &=  \int_a^b \bigg (\int_c^d f(x,y)dy \bigg )dx
\end{align}





\end{document}
